\documentclass[a4paper, 12pt, titlepage]{scrartcl}
\usepackage{authblk}

%============================================ Define Titlepage =====================================================%
     
\author[*]{\large Andreas Hill}
\author[ ]{\large Henning Buddenbaum}
\author[ ]{\large Daniel Mandallaz}
\affil[*]{\small Corresponding author: \underline{andreas.hill@usys.ethz.ch}}
\renewcommand\Authands{, } 

\date{\vspace{-0.8ex}}
\title{\LARGE Combining canopy height and tree species information for large scale timber volume estimations under strong heterogeneity of auxiliary data and variable sample plot sizes \vspace{-1ex}}
\subject{\vspace{-1.5cm} European Journal of Forest Research \\ \vspace{0.2cm} Supplementary Material \vspace{0.2cm} \vspace{-0.5cm}}


%============================================ Load Packages ========================================================%
\usepackage{geometry}
 \geometry{
 a4paper,
 left=20mm,
 top=25mm,
 bottom=20mm
 }
\usepackage{hyperref}
\usepackage[english]{babel}
\usepackage[utf8]{inputenc}
\usepackage[pdftex]{graphicx}
\usepackage{fancyhdr}
\usepackage{booktabs}
\usepackage{array} 
\usepackage{float}
\usepackage{tabularx}
\usepackage[explicit]{titlesec}
\usepackage{listings}
\usepackage{color}
\usepackage[labelfont=bf]{caption}
\usepackage[font=footnotesize]{caption}
\usepackage[font=footnotesize]{subcaption}
\usepackage{rotating}
\usepackage{longtable} 
 

 
%============================================ Define Settings ========================================================%

\definecolor{mygreen}{rgb}{0,0.6,0}
\definecolor{mygray}{rgb}{0.5,0.5,0.5}
\definecolor{mymauve}{rgb}{0.2,0.7,0.25}


\lstset{
  backgroundcolor=\color{white},  
  basicstyle=\footnotesize,        
  breakatwhitespace=false,       
  breaklines=true,                
  captionpos=b,                    
  commentstyle=\color{mygreen},    
  deletekeywords={...},           
  escapeinside={\%*}{*)},         
  extendedchars=true,                         
  keepspaces=true,                 
  keywordstyle=\color{blue},       
  language=Octave,                
  morekeywords={*,...},            
  numbers=left,                    
  numbersep=5pt,                   
  numberstyle=\tiny\color{mygray}, 
  rulecolor=\color{black},        
  showspaces=false,                
  showstringspaces=false,          
  showtabs=false,                 
  stepnumber=2,                   
  stringstyle=\color{mymauve},     
  tabsize=2,                      
    title=\lstname                 
}

\titlespacing{\section}{50pt}{1em}{1em}
\titlespacing{\subsection}{14pt}{3em}{1.5em}
\titlespacing{\subsubsection}{12pt}{2em}{1em}

\setlength{\parindent}{1em}

%============================================ Define personal commands=============================================%

% reocurring terms:
\newcommand{\bwi}{BWI3}
\newcommand{\adjrsq}{adjusted $R^2$}
\newcommand{\rmsecv}{RMSE$_{cv}$}
\newcommand{\mha}{m$^3$/ha}

%============================================ Start Document =======================================================%

\begin{document}

\maketitle 
\newpage

%============================================ SetUp directories ====================================================%

\pagestyle{fancy}
\fancyfoot[C]{\thepage}
\setlength{\headsep}{10mm}

\pagenumbering{arabic}
\fancyhf{}
\lhead{\footnotesize Hill et al. 2017: Combining canopy height and tree species information (Supplementary Material)}
\rhead{\footnotesize \thepage}


%============================================ Suppl. Material ========================================================%

% ----------------------------------------------------------------------- %
% 1
% ----------------------------------------------------------------------- %

\textbf{\large 1: Timber volume - height relationship on single tree level}

\vfill
\begin{figure}[h]
\centering
\subcaptionbox{Timber volume vs. tree height}{
  \includegraphics[width=0.6\textwidth]{fig/volr_vs_height_trees.png}%
  }\par\medskip
\subcaptionbox{Timber volume of sample trees plotted against tree height and tree species used in the \bwi{} taper functions}{
  \includegraphics[width=0.7\textwidth]{fig/volr_vs_hoehe_ba_tarif.png}
  }\par\medskip
\caption{Timber volume relationships on single tree level of all \bwi{} sample trees within RLP}
\label{fig:rel_terr}
\end{figure}
\vfill

\newpage

% ----------------------------------------------------------------------- %
% 2 
% ----------------------------------------------------------------------- %


\textbf{\large 2: Timber volume on plot level vs. predictor variables}

\begin{figure}[H]
\centering
\subcaptionbox{Timber volume on plot level vs. LiDAR \textit{meanheight} stratified by the error-free \textit{treespecies} variable}{
  \includegraphics[width=0.8\textwidth]{fig/true_species.png}%
  }\par\medskip
\subcaptionbox{Timber volume on plot level vs. LiDAR \textit{meanheight} stratified by calibrated \textit{treespecies} variable}{
  \includegraphics[width=0.8\textwidth]{fig/calibrated_species.png}
  }\par\medskip
\caption{Timber volume on sample plot level stratified by the \textit{lidaryears} and \textit{treespecies}}
\label{fig:tvol_vs_lyear_tspec}
\end{figure}


% ----------------------------------------------------------------------- %
% 3
% ----------------------------------------------------------------------- %

\textbf{\large 3: Classification accuracies of $treespecies$ variable}
% latex table generated in R 3.4.0 by xtable 1.8-2 package
% Thu Jul 13 18:18:15 2017
\begingroup\fontsize{9pt}{10pt}\selectfont
\begin{longtable}{lllrrrrrrr}
\caption{User's accuracies realized under various support choices for deriving 
                          the major tree species of a sample location. $class$: major tree species class of 
                          sample plot,
                          $prod.acc$: producer's accuarcy, $use.acc$: user's accuracy, $oaa$: overall accuracy,
                          $prod.acc_{cal}$: producer's accuarcy after calibration (use.acc and oaa respectively),
                          $n.ref$: number of validation data per tree species.}\\ \\
                          \hline
class & support & threshold & prod.acc & $prod.acc_{cal}$ & use.acc & $use.acc_{cal}$ & oaa & $oaa_{cal}$ & n.ref \\ 
  \hline
\endfirsthead
\hline
class & support & threshold & prod.acc & $prod.acc_{cal}$ & use.acc & $use.acc_{cal}$ & oaa & $oaa_{cal}$ & n.ref \\ 
  \hline
\endhead
\hline
\multicolumn{10}{l}{\footnotesize Continued on next page}
\endfoot
\endlastfoot
Beech & ind & 0\% & 50.77 & 70.64 & 66.69 & 58.31 & 52.25 & 57.14 & 1873 \\ 
  Douglas Fir & ind & 0\% & 53.63 & 41.60 & 43.23 & 51.88 & 52.25 & 57.14 & 399 \\ 
  Oak & ind & 0\% & 65.84 & 41.87 & 42.30 & 48.96 & 52.25 & 57.14 & 843 \\ 
  Spruce & ind & 0\% & 58.60 & 68.88 & 64.89 & 61.55 & 52.25 & 57.14 & 1041 \\ 
  Scots pine & ind & 0\% & 69.08 & 62.18 & 45.82 & 60.16 & 52.25 & 57.14 & 595 \\ 
  Mixed & ind & 0\% & 3.23 & 16.51 & 8.17 & 46.28 & 52.25 & 57.14 & 527 \\ 
  Beech & ind & 50\% & 50.95 & 69.24 & 65.00 & 56.82 & 50.63 & 54.60 & 1739 \\ 
  Douglas Fir & ind & 50\% & 55.35 & 45.72 & 44.52 & 49.85 & 50.63 & 54.60 & 374 \\ 
  Oak & ind & 50\% & 66.79 & 43.40 & 41.52 & 47.92 & 50.63 & 54.60 & 795 \\ 
  Spruce & ind & 50\% & 58.84 & 69.28 & 64.54 & 60.90 & 50.63 & 54.60 & 996 \\ 
  Scots pine & ind & 50\% & 69.34 & 60.77 & 45.13 & 59.46 & 50.63 & 54.60 & 548 \\ 
  Mixed & ind & 50\% & 9.93 & 16.83 & 19.48 & 34.49 & 50.63 & 54.60 & 826 \\ 
  Beech & ind & 60\% & 48.63 & 69.64 & 62.97 & 54.67 & 48.07 & 53.20 & 1647 \\ 
  Douglas Fir & ind & 60\% & 53.44 & 44.90 & 45.01 & 49.10 & 48.07 & 53.20 & 363 \\ 
  Oak & ind & 60\% & 64.44 & 42.51 & 40.33 & 47.89 & 48.07 & 53.20 & 748 \\ 
  Spruce & ind & 60\% & 57.56 & 69.57 & 64.28 & 62.75 & 48.07 & 53.20 & 966 \\ 
  Scots pine & ind & 60\% & 67.07 & 58.94 & 42.86 & 57.43 & 48.07 & 53.20 & 492 \\ 
  Mixed & ind & 60\% & 16.38 & 20.53 & 23.36 & 35.86 & 48.07 & 53.20 & 1062 \\ 
  Beech & ind & 80\% & 42.95 & 38.36 & 51.26 & 52.92 & 46.04 & 54.72 & 1134 \\ 
  Douglas Fir & ind & 80\% & 46.00 & 37.33 & 43.53 & 49.56 & 46.04 & 54.72 & 300 \\ 
  Oak & ind & 80\% & 64.19 & 21.86 & 29.36 & 40.34 & 46.04 & 54.72 & 430 \\ 
  Spruce & ind & 80\% & 52.78 & 61.36 & 61.38 & 65.06 & 46.04 & 54.72 & 792 \\ 
  Scots pine & ind & 80\% & 64.75 & 43.88 & 31.09 & 56.74 & 46.04 & 54.72 & 278 \\ 
  Mixed & ind & 80\% & 39.72 & 69.92 & 51.41 & 54.00 & 46.04 & 54.72 & 2344 \\ 
  Beech & ind & 100\% & 37.42 & 18.29 & 42.78 & 48.10 & 52.18 & 63.45 & 831 \\ 
  Douglas Fir & ind & 100\% & 42.98 & 22.37 & 37.69 & 38.64 & 52.18 & 63.45 & 228 \\ 
  Oak & ind & 100\% & 61.09 & 13.45 & 23.43 & 43.02 & 52.18 & 63.45 & 275 \\ 
  Spruce & ind & 100\% & 49.38 & 52.66 & 55.93 & 63.83 & 52.18 & 63.45 & 640 \\ 
  Scots pine & ind & 100\% & 62.01 & 28.49 & 24.78 & 58.62 & 52.18 & 63.45 & 179 \\ 
  Mixed & ind & 100\% & 56.00 & 87.07 & 68.33 & 65.90 & 52.18 & 63.45 & 3125 \\ 
  Beech & q25 & 0\% & 50.18 & 72.59 & 67.44 & 57.54 & 51.92 & 57.14 & 1923 \\ 
  Douglas Fir & q25 & 0\% & 54.61 & 43.39 & 42.77 & 50.73 & 51.92 & 57.14 & 401 \\ 
  Oak & q25 & 0\% & 65.23 & 39.91 & 41.59 & 50.52 & 51.92 & 57.14 & 857 \\ 
  Spruce & q25 & 0\% & 58.71 & 68.79 & 64.07 & 62.71 & 51.92 & 57.14 & 1051 \\ 
  Scots pine & q25 & 0\% & 68.52 & 60.94 & 44.97 & 61.46 & 51.92 & 57.14 & 594 \\ 
  Mixed & q25 & 0\% & 3.17 & 12.50 & 8.21 & 38.51 & 51.92 & 57.14 & 536 \\ 
  Beech & q25 & 50\% & 50.67 & 71.31 & 65.56 & 55.85 & 50.24 & 54.66 & 1788 \\ 
  Douglas Fir & q25 & 50\% & 55.05 & 46.54 & 43.58 & 52.24 & 50.24 & 54.66 & 376 \\ 
  Oak & q25 & 50\% & 65.64 & 39.93 & 40.60 & 48.57 & 50.24 & 54.66 & 809 \\ 
  Spruce & q25 & 50\% & 59.05 & 69.48 & 64.36 & 62.86 & 50.24 & 54.66 & 1006 \\ 
  Scots pine & q25 & 50\% & 68.56 & 60.33 & 44.22 & 60.11 & 50.24 & 54.66 & 547 \\ 
  Mixed & q25 & 50\% & 9.69 & 15.43 & 19.01 & 30.86 & 50.24 & 54.66 & 836 \\ 
  Beech & q25 & 60\% & 48.32 & 70.89 & 63.37 & 54.12 & 47.67 & 52.82 & 1697 \\ 
  Douglas Fir & q25 & 60\% & 52.60 & 45.48 & 43.94 & 49.85 & 47.67 & 52.82 & 365 \\ 
  Oak & q25 & 60\% & 64.04 & 39.50 & 39.10 & 46.52 & 47.67 & 52.82 & 762 \\ 
  Spruce & q25 & 60\% & 58.20 & 69.36 & 64.91 & 62.51 & 47.67 & 52.82 & 976 \\ 
  Scots pine & q25 & 60\% & 67.21 & 56.82 & 42.09 & 59.62 & 47.67 & 52.82 & 491 \\ 
  Mixed & q25 & 60\% & 14.75 & 19.23 & 21.82 & 33.88 & 47.67 & 52.82 & 1071 \\ 
  Beech & q25 & 80\% & 42.27 & 38.80 & 51.55 & 49.14 & 44.87 & 51.96 & 1183 \\ 
  Douglas Fir & q25 & 80\% & 45.36 & 35.10 & 46.13 & 50.96 & 44.87 & 51.96 & 302 \\ 
  Oak & q25 & 80\% & 61.09 & 20.59 & 28.01 & 36.84 & 44.87 & 51.96 & 442 \\ 
  Spruce & q25 & 80\% & 52.43 & 62.55 & 60.69 & 63.26 & 44.87 & 51.96 & 801 \\ 
  Scots pine & q25 & 80\% & 63.67 & 38.85 & 30.62 & 56.84 & 44.87 & 51.96 & 278 \\ 
  Mixed & q25 & 80\% & 38.29 & 64.56 & 48.47 & 50.85 & 44.87 & 51.96 & 2356 \\ 
  Beech & q25 & 100\% & 35.27 & 15.93 & 41.89 & 45.90 & 49.72 & 61.00 & 879 \\ 
  Douglas Fir & q25 & 100\% & 40.43 & 21.74 & 41.15 & 42.37 & 49.72 & 61.00 & 230 \\ 
  Oak & q25 & 100\% & 54.36 & 7.32 & 21.08 & 37.50 & 49.72 & 61.00 & 287 \\ 
  Spruce & q25 & 100\% & 49.77 & 52.85 & 56.67 & 59.86 & 49.72 & 61.00 & 649 \\ 
  Scots pine & q25 & 100\% & 59.78 & 19.55 & 23.83 & 44.30 & 49.72 & 61.00 & 179 \\ 
  Mixed & q25 & 100\% & 53.44 & 85.47 & 63.59 & 63.39 & 49.72 & 61.00 & 3138 \\ 
  Beech & q50 & 0\% & 51.14 & 73.60 & 67.53 & 59.03 & 53.06 & 58.45 & 1932 \\ 
  Douglas Fir & q50 & 0\% & 54.59 & 45.16 & 42.23 & 55.49 & 53.06 & 58.45 & 403 \\ 
  Oak & q50 & 0\% & 68.29 & 41.93 & 41.79 & 51.94 & 53.06 & 58.45 & 861 \\ 
  Spruce & q50 & 0\% & 59.92 & 70.47 & 64.65 & 63.26 & 53.06 & 58.45 & 1053 \\ 
  Scots pine & q50 & 0\% & 70.59 & 61.68 & 45.16 & 61.06 & 53.06 & 58.45 & 595 \\ 
  Mixed & q50 & 0\% & 1.85 & 13.52 & 11.49 & 41.01 & 53.06 & 58.45 & 540 \\ 
  Beech & q50 & 50\% & 51.64 & 72.06 & 66.10 & 57.58 & 51.32 & 55.76 & 1797 \\ 
  Douglas Fir & q50 & 50\% & 56.08 & 47.62 & 43.80 & 54.38 & 51.32 & 55.76 & 378 \\ 
  Oak & q50 & 50\% & 68.39 & 44.90 & 41.31 & 51.41 & 51.32 & 55.76 & 813 \\ 
  Spruce & q50 & 50\% & 60.22 & 69.74 & 64.92 & 63.05 & 51.32 & 55.76 & 1008 \\ 
  Scots pine & q50 & 50\% & 70.26 & 60.22 & 44.87 & 60.22 & 51.32 & 55.76 & 548 \\ 
  Mixed & q50 & 50\% & 8.93 & 15.36 & 21.01 & 29.93 & 51.32 & 55.76 & 840 \\ 
  Beech & q50 & 60\% & 48.21 & 70.26 & 64.42 & 56.17 & 48.76 & 53.44 & 1705 \\ 
  Douglas Fir & q50 & 60\% & 52.86 & 43.60 & 46.08 & 52.12 & 48.76 & 53.44 & 367 \\ 
  Oak & q50 & 60\% & 64.75 & 45.17 & 40.66 & 49.50 & 48.76 & 53.44 & 766 \\ 
  Spruce & q50 & 60\% & 58.69 & 69.73 & 65.98 & 62.34 & 48.76 & 53.44 & 978 \\ 
  Scots pine & q50 & 60\% & 67.07 & 55.28 & 43.54 & 59.65 & 48.76 & 53.44 & 492 \\ 
  Mixed & q50 & 60\% & 19.42 & 20.35 & 24.91 & 31.51 & 48.76 & 53.44 & 1076 \\ 
  Beech & q50 & 80\% & 41.21 & 44.91 & 54.08 & 53.78 & 47.33 & 54.59 & 1189 \\ 
  Douglas Fir & q50 & 80\% & 44.41 & 37.50 & 47.87 & 56.16 & 47.33 & 54.59 & 304 \\ 
  Oak & q50 & 80\% & 61.88 & 24.89 & 30.53 & 39.64 & 47.33 & 54.59 & 446 \\ 
  Spruce & q50 & 80\% & 52.30 & 63.76 & 63.35 & 64.81 & 47.33 & 54.59 & 803 \\ 
  Scots pine & q50 & 80\% & 63.67 & 39.57 & 32.90 & 58.20 & 47.33 & 54.59 & 278 \\ 
  Mixed & q50 & 80\% & 44.42 & 65.91 & 50.22 & 53.19 & 47.33 & 54.59 & 2364 \\ 
  Beech & q50 & 100\% & 29.56 & 20.16 & 46.11 & 46.35 & 54.83 & 62.26 & 883 \\ 
  Douglas Fir & q50 & 100\% & 33.04 & 23.91 & 47.50 & 49.55 & 54.83 & 62.26 & 230 \\ 
  Oak & q50 & 100\% & 47.75 & 10.38 & 24.34 & 46.88 & 54.83 & 62.26 & 289 \\ 
  Spruce & q50 & 100\% & 43.63 & 56.99 & 60.81 & 61.94 & 54.83 & 62.26 & 651 \\ 
  Scots pine & q50 & 100\% & 55.87 & 20.67 & 29.59 & 49.33 & 54.83 & 62.26 & 179 \\ 
  Mixed & q50 & 100\% & 66.40 & 85.06 & 63.69 & 64.59 & 54.83 & 62.26 & 3152 \\ 
  Beech & q80 & 0\% & 51.91 & 73.58 & 67.25 & 60.45 & 53.98 & 59.64 & 1938 \\ 
  Douglas Fir & q80 & 0\% & 55.91 & 47.04 & 42.99 & 55.52 & 53.98 & 59.64 & 406 \\ 
  Oak & q80 & 0\% & 68.55 & 46.82 & 42.30 & 53.01 & 53.98 & 59.64 & 865 \\ 
  Spruce & q80 & 0\% & 61.90 & 69.67 & 66.03 & 65.51 & 53.98 & 59.64 & 1055 \\ 
  Scots pine & q80 & 0\% & 71.74 & 64.21 & 46.48 & 61.15 & 53.98 & 59.64 & 598 \\ 
  Mixed & q80 & 0\% & 1.84 & 15.26 & 14.71 & 43.92 & 53.98 & 59.64 & 544 \\ 
  Beech & q80 & 50\% & 50.69 & 72.49 & 66.18 & 58.37 & 51.72 & 57.16 & 1803 \\ 
  Douglas Fir & q80 & 50\% & 54.86 & 45.67 & 46.04 & 53.54 & 51.72 & 57.16 & 381 \\ 
  Oak & q80 & 50\% & 67.93 & 48.84 & 42.43 & 54.21 & 51.72 & 57.16 & 817 \\ 
  Spruce & q80 & 50\% & 60.79 & 71.19 & 66.81 & 64.14 & 51.72 & 57.16 & 1010 \\ 
  Scots pine & q80 & 50\% & 71.58 & 61.02 & 47.41 & 59.71 & 51.72 & 57.16 & 549 \\ 
  Mixed & q80 & 50\% & 13.12 & 18.44 & 21.55 & 36.79 & 51.72 & 57.16 & 846 \\ 
  Beech & q80 & 60\% & 46.05 & 71.07 & 66.39 & 56.51 & 48.85 & 55.07 & 1711 \\ 
  Douglas Fir & q80 & 60\% & 49.73 & 47.30 & 50.41 & 55.03 & 48.85 & 55.07 & 370 \\ 
  Oak & q80 & 60\% & 63.90 & 48.18 & 42.75 & 52.33 & 48.85 & 55.07 & 770 \\ 
  Spruce & q80 & 60\% & 56.53 & 70.41 & 68.23 & 64.97 & 48.85 & 55.07 & 980 \\ 
  Scots pine & q80 & 60\% & 66.73 & 58.62 & 47.07 & 58.86 & 48.85 & 55.07 & 493 \\ 
  Mixed & q80 & 60\% & 27.17 & 21.81 & 24.66 & 35.01 & 48.85 & 55.07 & 1082 \\ 
  Beech & q80 & 80\% & 36.77 & 40.37 & 56.65 & 52.05 & 48.69 & 54.37 & 1194 \\ 
  Douglas Fir & q80 & 80\% & 39.74 & 37.46 & 54.71 & 60.53 & 48.69 & 54.37 & 307 \\ 
  Oak & q80 & 80\% & 56.47 & 24.55 & 31.59 & 41.83 & 48.69 & 54.37 & 448 \\ 
  Spruce & q80 & 80\% & 48.51 & 64.43 & 65.66 & 64.19 & 48.69 & 54.37 & 804 \\ 
  Scots pine & q80 & 80\% & 60.57 & 43.73 & 36.98 & 62.89 & 48.69 & 54.37 & 279 \\ 
  Mixed & q80 & 80\% & 53.03 & 67.06 & 49.26 & 52.61 & 48.69 & 54.37 & 2374 \\ 
  Beech & q80 & 100\% & 21.56 & 20.32 & 49.35 & 46.15 & 57.92 & 61.52 & 886 \\ 
  Douglas Fir & q80 & 100\% & 25.54 & 22.51 & 55.14 & 46.43 & 57.92 & 61.52 & 231 \\ 
  Oak & q80 & 100\% & 34.48 & 12.76 & 26.46 & 41.57 & 57.92 & 61.52 & 290 \\ 
  Spruce & q80 & 100\% & 32.36 & 50.46 & 62.61 & 59.17 & 57.92 & 61.52 & 652 \\ 
  Scots pine & q80 & 100\% & 44.44 & 25.00 & 35.56 & 53.57 & 57.92 & 61.52 & 180 \\ 
  Mixed & q80 & 100\% & 78.62 & 84.72 & 62.69 & 64.26 & 57.92 & 61.52 & 3167 \\ 
  Beech & q100 & 0\% & 53.72 & 70.93 & 66.92 & 57.90 & 55.03 & 57.27 & 1947 \\ 
  Douglas Fir & q100 & 0\% & 54.77 & 46.94 & 46.28 & 54.24 & 55.03 & 57.27 & 409 \\ 
  Oak & q100 & 0\% & 70.92 & 45.75 & 43.06 & 51.49 & 55.03 & 57.27 & 870 \\ 
  Spruce & q100 & 0\% & 62.30 & 66.07 & 64.87 & 62.20 & 55.03 & 57.27 & 1061 \\ 
  Scots pine & q100 & 0\% & 73.88 & 64.89 & 47.64 & 62.30 & 55.03 & 57.27 & 601 \\ 
  Mixed & q100 & 0\% & 0.36 & 9.76 & 20.00 & 30.68 & 55.03 & 57.27 & 553 \\ 
  Beech & q100 & 50\% & 46.96 & 69.43 & 68.35 & 56.39 & 51.08 & 54.88 & 1812 \\ 
  Douglas Fir & q100 & 50\% & 48.30 & 49.35 & 55.06 & 56.08 & 51.08 & 54.88 & 383 \\ 
  Oak & q100 & 50\% & 66.42 & 47.20 & 45.92 & 53.08 & 51.08 & 54.88 & 822 \\ 
  Spruce & q100 & 50\% & 56.00 & 67.42 & 68.14 & 60.67 & 51.08 & 54.88 & 1016 \\ 
  Scots pine & q100 & 50\% & 67.57 & 63.22 & 56.26 & 61.12 & 51.08 & 54.88 & 552 \\ 
  Mixed & q100 & 50\% & 29.79 & 13.67 & 21.74 & 26.47 & 51.08 & 54.88 & 856 \\ 
  Beech & q100 & 60\% & 39.48 & 69.30 & 69.50 & 55.86 & 47.86 & 53.30 & 1720 \\ 
  Douglas Fir & q100 & 60\% & 39.78 & 44.35 & 61.41 & 52.22 & 47.86 & 53.30 & 372 \\ 
  Oak & q100 & 60\% & 58.45 & 46.84 & 47.99 & 52.23 & 47.86 & 53.30 & 775 \\ 
  Spruce & q100 & 60\% & 49.49 & 66.33 & 70.62 & 61.29 & 47.86 & 53.30 & 986 \\ 
  Scots pine & q100 & 60\% & 62.50 & 60.08 & 59.96 & 60.82 & 47.86 & 53.30 & 496 \\ 
  Mixed & q100 & 60\% & 48.17 & 20.88 & 25.40 & 30.85 & 47.86 & 53.30 & 1092 \\ 
  Beech & q100 & 80\% & 25.92 & 40.67 & 64.39 & 51.64 & 51.02 & 51.87 & 1200 \\ 
  Douglas Fir & q100 & 80\% & 23.62 & 27.18 & 67.59 & 56.00 & 51.02 & 51.87 & 309 \\ 
  Oak & q100 & 80\% & 44.15 & 24.06 & 37.04 & 40.82 & 51.02 & 51.87 & 453 \\ 
  Spruce & q100 & 80\% & 33.95 & 56.01 & 69.54 & 59.01 & 51.02 & 51.87 & 807 \\ 
  Scots pine & q100 & 80\% & 55.36 & 42.86 & 53.82 & 55.81 & 51.02 & 51.87 & 280 \\ 
  Mixed & q100 & 80\% & 73.70 & 65.59 & 48.59 & 50.65 & 51.02 & 51.87 & 2392 \\ 
  Beech & q100 & 100\% & 4.15 & 17.51 & 55.22 & 40.52 & 59.42 & 59.22 & 891 \\ 
  Douglas Fir & q100 & 100\% & 5.60 & 8.62 & 68.42 & 33.33 & 59.42 & 59.22 & 232 \\ 
  Oak & q100 & 100\% & 8.19 & 8.87 & 25.53 & 30.23 & 59.42 & 59.22 & 293 \\ 
  Spruce & q100 & 100\% & 10.28 & 41.26 & 80.72 & 52.75 & 59.42 & 59.22 & 652 \\ 
  Scots pine & q100 & 100\% & 20.99 & 28.73 & 58.46 & 54.17 & 59.42 & 59.22 & 181 \\ 
  Mixed & q100 & 100\% & 95.68 & 84.56 & 59.73 & 62.71 & 59.42 & 59.22 & 3192 \\ 
  \hline
\end{longtable}
\endgroup
\newpage



% ----------------------------------------------------------------------- %
% 4 & 5
% ----------------------------------------------------------------------- %

\textbf{\large 4: Model accuracies}\\

% latex table generated in R 3.4.2 by xtable 1.8-2 package
% Sat Jan 06 16:02:43 2018
\begingroup\fontsize{7pt}{8pt}\selectfont
\begin{longtable}{lllrrrrrrrr}
	\caption{Model accuracies realized under various 
		support choices for the CHM- and \textbf{uncalibrated} $treespecies$ 
		explanatory variables}\\ \\
	\hline
    $support_{chm}$ & $support_{tspec}$ & $threshold$ & $R^2_{adj}$ & $rmse_{cv}$ & $RMSE_{cv}[\%]$ & $AIC$ & $R^2_{adj, ref}$ & $RMSE_{cv, ref}$ & $RMSE_{cv, ref}[\%]$ & $AIC_{ref}$ \\ 
	\hline
	\endfirsthead
	\hline
    $support_{chm}$ & $support_{tspec}$ & $threshold$ & $R^2_{adj}$ & $RMSE_{cv}$ & $RMSE_{cv}[\%]$ & $AIC$ & $R^2_{adj, ref}$ & $RMSE_{cv, ref}$ & $RMSE_{cv, ref}[\%]$ & $AIC_{ref}$ \\ 
	\hline
	\endhead
	\hline
	\multicolumn{11}{l}{\footnotesize Continued on next page}
	\endfoot
	\endlastfoot
ind & ind & 0\% & 0.44 & 142.03 & 44.20 & 64408.57 & 0.45 & 139.71 & 43.48 & 64256.93 \\ 
  ind & ind & 50\% & 0.44 & 141.74 & 44.11 & 64401.58 & 0.45 & 139.85 & 43.52 & 64268.78 \\ 
  ind & ind & 60\% & 0.44 & 141.95 & 44.18 & 64405.91 & 0.45 & 139.48 & 43.41 & 64255.23 \\ 
  ind & ind & 80\% & 0.44 & 141.68 & 44.09 & 64384.81 & 0.46 & 138.92 & 43.23 & 64225.80 \\ 
  ind & ind & 100\% & 0.44 & 141.97 & 44.18 & 64385.38 & 0.46 & 138.85 & 43.21 & 64204.65 \\ 
  ind & q25 & 0\% & 0.44 & 140.67 & 44.21 & 64927.75 & 0.46 & 139.25 & 43.77 & 64780.23 \\ 
  ind & q25 & 50\% & 0.44 & 140.56 & 44.18 & 64912.21 & 0.46 & 139.37 & 43.81 & 64795.25 \\ 
  ind & q25 & 60\% & 0.44 & 140.63 & 44.20 & 64914.25 & 0.46 & 139.23 & 43.76 & 64783.04 \\ 
  ind & q25 & 80\% & 0.45 & 140.77 & 44.25 & 64906.60 & 0.46 & 138.57 & 43.56 & 64749.77 \\ 
  ind & q25 & 100\% & 0.44 & 140.72 & 44.23 & 64909.56 & 0.46 & 138.14 & 43.42 & 64725.81 \\ 
  ind & q50 & 0\% & 0.44 & 141.62 & 44.71 & 65195.36 & 0.46 & 139.36 & 44.00 & 65057.03 \\ 
  ind & q50 & 50\% & 0.44 & 141.58 & 44.70 & 65186.32 & 0.46 & 139.60 & 44.07 & 65072.06 \\ 
  ind & q50 & 60\% & 0.45 & 141.55 & 44.69 & 65184.31 & 0.46 & 139.53 & 44.05 & 65060.08 \\ 
  ind & q50 & 80\% & 0.44 & 141.61 & 44.70 & 65197.21 & 0.46 & 138.90 & 43.85 & 65026.35 \\ 
  ind & q50 & 100\% & 0.44 & 142.05 & 44.85 & 65230.07 & 0.46 & 138.60 & 43.75 & 65002.79 \\ 
  ind & q80 & 0\% & 0.45 & 140.73 & 44.58 & 65392.45 & 0.46 & 139.43 & 44.16 & 65282.95 \\ 
  ind & q80 & 50\% & 0.45 & 140.75 & 44.58 & 65391.12 & 0.46 & 139.71 & 44.25 & 65297.81 \\ 
  ind & q80 & 60\% & 0.44 & 141.25 & 44.74 & 65414.38 & 0.46 & 139.58 & 44.21 & 65286.72 \\ 
  ind & q80 & 80\% & 0.45 & 141.20 & 44.72 & 65397.48 & 0.46 & 138.97 & 44.02 & 65252.00 \\ 
  ind & q80 & 100\% & 0.44 & 142.03 & 44.98 & 65482.59 & 0.46 & 138.84 & 43.98 & 65227.79 \\ 
  ind & q100 & 0\% & 0.45 & 140.81 & 44.34 & 65701.65 & 0.46 & 138.83 & 43.72 & 65570.10 \\ 
  ind & q100 & 50\% & 0.45 & 140.58 & 44.27 & 65688.26 & 0.46 & 139.13 & 43.81 & 65585.38 \\ 
  ind & q100 & 60\% & 0.44 & 140.87 & 44.36 & 65708.90 & 0.46 & 139.14 & 43.81 & 65575.20 \\ 
  ind & q100 & 80\% & 0.44 & 141.69 & 44.62 & 65772.71 & 0.46 & 138.39 & 43.58 & 65541.28 \\ 
  ind & q100 & 100\% & 0.42 & 143.37 & 45.14 & 65885.80 & 0.46 & 138.25 & 43.53 & 65513.09 \\ 
  q25 & ind & 0\% & 0.44 & 142.57 & 44.60 & 64938.10 & 0.46 & 140.09 & 43.82 & 64776.64 \\ 
  q25 & ind & 50\% & 0.44 & 142.53 & 44.59 & 64935.55 & 0.46 & 140.28 & 43.88 & 64790.59 \\ 
  q25 & ind & 60\% & 0.44 & 142.64 & 44.62 & 64936.98 & 0.46 & 139.96 & 43.78 & 64778.10 \\ 
  q25 & ind & 80\% & 0.44 & 142.39 & 44.54 & 64908.14 & 0.46 & 139.43 & 43.62 & 64756.67 \\ 
  q25 & ind & 100\% & 0.44 & 142.06 & 44.44 & 64895.64 & 0.46 & 138.92 & 43.46 & 64729.99 \\ 
  q25 & q25 & 0\% & 0.45 & 141.99 & 45.44 & 65965.52 & 0.47 & 139.91 & 44.78 & 65808.37 \\ 
  q25 & q25 & 50\% & 0.45 & 141.87 & 45.40 & 65955.14 & 0.47 & 140.07 & 44.82 & 65822.64 \\ 
  q25 & q25 & 60\% & 0.45 & 141.89 & 45.41 & 65957.07 & 0.47 & 139.77 & 44.73 & 65809.39 \\ 
  q25 & q25 & 80\% & 0.45 & 141.91 & 45.42 & 65944.31 & 0.47 & 139.33 & 44.59 & 65783.19 \\ 
  q25 & q25 & 100\% & 0.45 & 142.05 & 45.46 & 65955.03 & 0.48 & 138.53 & 44.33 & 65747.03 \\ 
  q25 & q50 & 0\% & 0.45 & 139.70 & 44.82 & 66237.98 & 0.47 & 138.15 & 44.33 & 66087.78 \\ 
  q25 & q50 & 50\% & 0.45 & 139.61 & 44.80 & 66229.45 & 0.47 & 138.31 & 44.38 & 66101.77 \\ 
  q25 & q50 & 60\% & 0.45 & 139.71 & 44.83 & 66226.29 & 0.47 & 138.09 & 44.31 & 66088.89 \\ 
  q25 & q50 & 80\% & 0.45 & 139.87 & 44.88 & 66241.23 & 0.47 & 137.55 & 44.13 & 66062.63 \\ 
  q25 & q50 & 100\% & 0.45 & 140.32 & 45.02 & 66280.18 & 0.47 & 136.83 & 43.90 & 66026.09 \\ 
  q25 & q80 & 0\% & 0.46 & 140.42 & 45.24 & 66429.54 & 0.47 & 139.34 & 44.89 & 66311.89 \\ 
  q25 & q80 & 50\% & 0.46 & 140.36 & 45.22 & 66432.75 & 0.47 & 139.72 & 45.02 & 66325.46 \\ 
  q25 & q80 & 60\% & 0.45 & 140.78 & 45.36 & 66453.60 & 0.47 & 139.52 & 44.95 & 66313.60 \\ 
  q25 & q80 & 80\% & 0.46 & 140.53 & 45.27 & 66439.86 & 0.47 & 139.07 & 44.80 & 66286.42 \\ 
  q25 & q80 & 100\% & 0.45 & 142.02 & 45.75 & 66532.00 & 0.48 & 138.48 & 44.61 & 66249.45 \\ 
  q25 & q100 & 0\% & 0.45 & 142.36 & 45.55 & 66767.76 & 0.47 & 140.09 & 44.83 & 66626.46 \\ 
  q25 & q100 & 50\% & 0.46 & 141.92 & 45.41 & 66755.71 & 0.47 & 140.32 & 44.90 & 66640.38 \\ 
  q25 & q100 & 60\% & 0.45 & 142.18 & 45.49 & 66783.49 & 0.47 & 140.42 & 44.93 & 66629.48 \\ 
  q25 & q100 & 80\% & 0.45 & 143.32 & 45.86 & 66854.10 & 0.47 & 140.16 & 44.85 & 66602.55 \\ 
  q25 & q100 & 100\% & 0.43 & 145.09 & 46.43 & 66974.26 & 0.48 & 139.97 & 44.79 & 66563.15 \\ 
  q50 & ind & 0\% & 0.45 & 141.43 & 44.24 & 64854.59 & 0.47 & 138.82 & 43.42 & 64686.53 \\ 
  q50 & ind & 50\% & 0.45 & 141.41 & 44.24 & 64850.86 & 0.47 & 138.93 & 43.46 & 64698.27 \\ 
  q50 & ind & 60\% & 0.45 & 141.50 & 44.27 & 64853.84 & 0.47 & 138.70 & 43.39 & 64689.17 \\ 
  q50 & ind & 80\% & 0.45 & 141.33 & 44.21 & 64827.37 & 0.47 & 138.27 & 43.25 & 64672.63 \\ 
  q50 & ind & 100\% & 0.45 & 141.16 & 44.16 & 64827.16 & 0.47 & 137.91 & 43.14 & 64652.47 \\ 
  q50 & q25 & 0\% & 0.46 & 140.61 & 45.00 & 65874.22 & 0.48 & 138.36 & 44.28 & 65710.42 \\ 
  q50 & q25 & 50\% & 0.46 & 140.45 & 44.95 & 65862.51 & 0.48 & 138.43 & 44.30 & 65722.03 \\ 
  q50 & q25 & 60\% & 0.46 & 140.49 & 44.96 & 65864.18 & 0.48 & 138.19 & 44.22 & 65711.95 \\ 
  q50 & q25 & 80\% & 0.46 & 140.59 & 44.99 & 65854.77 & 0.48 & 137.97 & 44.15 & 65691.89 \\ 
  q50 & q25 & 100\% & 0.46 & 140.58 & 44.99 & 65860.14 & 0.48 & 137.41 & 43.98 & 65664.81 \\ 
  q50 & q50 & 0\% & 0.46 & 138.57 & 44.46 & 66145.64 & 0.48 & 136.89 & 43.92 & 65989.18 \\ 
  q50 & q50 & 50\% & 0.46 & 138.52 & 44.44 & 66138.78 & 0.48 & 137.05 & 43.97 & 66001.00 \\ 
  q50 & q50 & 60\% & 0.46 & 138.62 & 44.48 & 66134.47 & 0.48 & 136.89 & 43.92 & 65991.20 \\ 
  q50 & q50 & 80\% & 0.46 & 138.78 & 44.53 & 66146.85 & 0.48 & 136.48 & 43.79 & 65971.13 \\ 
  q50 & q50 & 100\% & 0.46 & 139.30 & 44.70 & 66189.99 & 0.48 & 135.98 & 43.63 & 65943.10 \\ 
  q50 & q80 & 0\% & 0.47 & 138.34 & 44.57 & 66333.05 & 0.48 & 137.22 & 44.21 & 66211.70 \\ 
  q50 & q80 & 50\% & 0.47 & 138.35 & 44.57 & 66341.81 & 0.48 & 137.60 & 44.33 & 66223.75 \\ 
  q50 & q80 & 60\% & 0.46 & 138.77 & 44.71 & 66361.90 & 0.48 & 137.43 & 44.28 & 66214.72 \\ 
  q50 & q80 & 80\% & 0.46 & 138.54 & 44.63 & 66347.78 & 0.48 & 137.11 & 44.17 & 66193.69 \\ 
  q50 & q80 & 100\% & 0.45 & 140.08 & 45.13 & 66444.51 & 0.48 & 136.63 & 44.02 & 66165.28 \\ 
  q50 & q100 & 0\% & 0.46 & 141.39 & 45.24 & 66681.45 & 0.48 & 138.94 & 44.46 & 66528.70 \\ 
  q50 & q100 & 50\% & 0.47 & 140.88 & 45.08 & 66665.99 & 0.48 & 139.14 & 44.52 & 66540.96 \\ 
  q50 & q100 & 60\% & 0.46 & 141.14 & 45.16 & 66693.59 & 0.48 & 139.27 & 44.56 & 66532.93 \\ 
  q50 & q100 & 80\% & 0.45 & 142.17 & 45.49 & 66764.60 & 0.48 & 139.15 & 44.52 & 66511.93 \\ 
  q50 & q100 & 100\% & 0.44 & 143.86 & 46.03 & 66879.02 & 0.48 & 138.81 & 44.42 & 66480.34 \\ 
  q80 & ind & 0\% & 0.45 & 141.92 & 44.40 & 64868.26 & 0.47 & 139.22 & 43.55 & 64695.13 \\ 
  q80 & ind & 50\% & 0.45 & 141.87 & 44.38 & 64862.40 & 0.47 & 139.41 & 43.61 & 64711.20 \\ 
  q80 & ind & 60\% & 0.45 & 142.00 & 44.42 & 64866.26 & 0.47 & 139.20 & 43.54 & 64701.77 \\ 
  q80 & ind & 80\% & 0.45 & 141.89 & 44.39 & 64847.05 & 0.47 & 138.77 & 43.41 & 64681.35 \\ 
  q80 & ind & 100\% & 0.45 & 141.79 & 44.35 & 64851.41 & 0.47 & 138.41 & 43.30 & 64658.70 \\ 
  q80 & q25 & 0\% & 0.46 & 140.65 & 45.01 & 65888.64 & 0.48 & 138.44 & 44.30 & 65720.92 \\ 
  q80 & q25 & 50\% & 0.46 & 140.50 & 44.96 & 65877.65 & 0.48 & 138.60 & 44.35 & 65737.10 \\ 
  q80 & q25 & 60\% & 0.46 & 140.54 & 44.98 & 65878.63 & 0.48 & 138.46 & 44.31 & 65726.72 \\ 
  q80 & q25 & 80\% & 0.46 & 140.65 & 45.01 & 65869.87 & 0.48 & 138.20 & 44.23 & 65703.95 \\ 
  q80 & q25 & 100\% & 0.46 & 140.70 & 45.03 & 65878.28 & 0.48 & 137.46 & 43.99 & 65674.03 \\ 
  q80 & q50 & 0\% & 0.46 & 138.78 & 44.53 & 66159.84 & 0.48 & 137.07 & 43.98 & 66000.20 \\ 
  q80 & q50 & 50\% & 0.46 & 138.74 & 44.51 & 66154.54 & 0.48 & 137.26 & 44.04 & 66016.68 \\ 
  q80 & q50 & 60\% & 0.46 & 138.88 & 44.56 & 66151.52 & 0.48 & 136.96 & 43.94 & 66005.75 \\ 
  q80 & q50 & 80\% & 0.46 & 139.03 & 44.61 & 66162.48 & 0.48 & 136.59 & 43.82 & 65982.82 \\ 
  q80 & q50 & 100\% & 0.46 & 139.62 & 44.80 & 66209.86 & 0.48 & 136.14 & 43.68 & 65952.40 \\ 
  q80 & q80 & 0\% & 0.46 & 138.03 & 44.47 & 66348.58 & 0.48 & 136.86 & 44.09 & 66222.79 \\ 
  q80 & q80 & 50\% & 0.46 & 138.08 & 44.48 & 66357.36 & 0.48 & 137.28 & 44.23 & 66239.71 \\ 
  q80 & q80 & 60\% & 0.46 & 138.50 & 44.62 & 66378.68 & 0.48 & 137.06 & 44.16 & 66229.36 \\ 
  q80 & q80 & 80\% & 0.46 & 138.41 & 44.59 & 66367.72 & 0.48 & 136.71 & 44.04 & 66205.47 \\ 
  q80 & q80 & 100\% & 0.45 & 139.94 & 45.09 & 66461.68 & 0.48 & 136.34 & 43.93 & 66174.95 \\ 
  q80 & q100 & 0\% & 0.46 & 141.52 & 45.28 & 66698.65 & 0.48 & 138.91 & 44.45 & 66542.06 \\ 
  q80 & q100 & 50\% & 0.46 & 140.95 & 45.10 & 66684.27 & 0.48 & 139.16 & 44.53 & 66558.80 \\ 
  q80 & q100 & 60\% & 0.46 & 141.31 & 45.22 & 66715.73 & 0.48 & 139.20 & 44.54 & 66550.13 \\ 
  q80 & q100 & 80\% & 0.45 & 142.29 & 45.53 & 66783.55 & 0.48 & 139.01 & 44.48 & 66527.02 \\ 
  q80 & q100 & 100\% & 0.44 & 143.91 & 46.05 & 66895.38 & 0.48 & 138.74 & 44.39 & 66492.22 \\ 
  q100 & ind & 0\% & 0.39 & 149.44 & 46.75 & 65391.23 & 0.41 & 147.18 & 46.04 & 65241.42 \\ 
  q100 & ind & 50\% & 0.39 & 149.34 & 46.72 & 65379.90 & 0.40 & 147.32 & 46.08 & 65260.46 \\ 
  q100 & ind & 60\% & 0.39 & 149.43 & 46.75 & 65382.54 & 0.41 & 147.10 & 46.01 & 65249.02 \\ 
  q100 & ind & 80\% & 0.39 & 149.50 & 46.77 & 65377.00 & 0.41 & 146.57 & 45.85 & 65224.48 \\ 
  q100 & ind & 100\% & 0.39 & 149.15 & 46.65 & 65366.75 & 0.41 & 146.01 & 45.67 & 65194.50 \\ 
  q100 & q25 & 0\% & 0.40 & 147.61 & 47.24 & 66442.67 & 0.42 & 145.59 & 46.59 & 66291.14 \\ 
  q100 & q25 & 50\% & 0.40 & 147.46 & 47.19 & 66430.07 & 0.41 & 145.73 & 46.63 & 66310.71 \\ 
  q100 & q25 & 60\% & 0.40 & 147.36 & 47.16 & 66428.18 & 0.42 & 145.51 & 46.56 & 66299.12 \\ 
  q100 & q25 & 80\% & 0.40 & 147.39 & 47.17 & 66416.52 & 0.42 & 145.21 & 46.47 & 66271.49 \\ 
  q100 & q25 & 100\% & 0.40 & 147.60 & 47.23 & 66426.82 & 0.42 & 144.46 & 46.23 & 66233.99 \\ 
  q100 & q50 & 0\% & 0.40 & 146.76 & 47.09 & 66720.28 & 0.42 & 145.07 & 46.54 & 66572.83 \\ 
  q100 & q50 & 50\% & 0.40 & 146.63 & 47.05 & 66712.95 & 0.41 & 145.32 & 46.62 & 66592.86 \\ 
  q100 & q50 & 60\% & 0.40 & 146.85 & 47.11 & 66708.05 & 0.41 & 145.14 & 46.57 & 66581.36 \\ 
  q100 & q50 & 80\% & 0.40 & 147.27 & 47.25 & 66722.92 & 0.42 & 144.59 & 46.39 & 66553.52 \\ 
  q100 & q50 & 100\% & 0.39 & 147.64 & 47.37 & 66759.50 & 0.42 & 144.08 & 46.23 & 66516.11 \\ 
  q100 & q80 & 0\% & 0.40 & 145.20 & 46.78 & 66917.50 & 0.42 & 144.07 & 46.42 & 66796.64 \\ 
  q100 & q80 & 50\% & 0.40 & 145.12 & 46.76 & 66917.72 & 0.41 & 144.47 & 46.55 & 66816.62 \\ 
  q100 & q80 & 60\% & 0.40 & 145.58 & 46.90 & 66935.84 & 0.42 & 144.29 & 46.49 & 66805.78 \\ 
  q100 & q80 & 80\% & 0.40 & 145.62 & 46.92 & 66932.29 & 0.42 & 143.97 & 46.39 & 66776.81 \\ 
  q100 & q80 & 100\% & 0.39 & 147.12 & 47.40 & 67015.72 & 0.42 & 143.54 & 46.25 & 66739.98 \\ 
  q100 & q100 & 0\% & 0.40 & 148.90 & 47.65 & 67272.02 & 0.42 & 146.01 & 46.72 & 67121.18 \\ 
  q100 & q100 & 50\% & 0.40 & 148.28 & 47.45 & 67258.42 & 0.41 & 146.33 & 46.82 & 67140.71 \\ 
  q100 & q100 & 60\% & 0.40 & 148.59 & 47.55 & 67283.11 & 0.42 & 146.45 & 46.86 & 67130.60 \\ 
  q100 & q100 & 80\% & 0.39 & 149.75 & 47.92 & 67353.11 & 0.42 & 146.13 & 46.76 & 67102.80 \\ 
  q100 & q100 & 100\% & 0.38 & 151.55 & 48.49 & 67469.16 & 0.42 & 145.80 & 46.66 & 67062.13 \\ 
  \hline
\end{longtable}
\endgroup\newpage

% latex table generated in R 3.3.2 by xtable 1.8-2 package
% Tue Apr 25 10:42:21 2017
\begingroup\fontsize{9pt}{10pt}\selectfont
\begin{longtable}{lllrrrrrr}
\caption{Model accuracies realized under various 
                       support choices for the CHM- and \textbf{calibrated} $treespecies$ 
                       explanatory variables}\\ \\
\hline
$support_{chm}$ & $support_{tspec}$ & threshold & $R^2_{adj}$ & $rmse_{cv}$ & AIC & $R^2_{adj, ref}$ & $rmse_{cv, ref}$ & $AIC_{ref}$ \\ 
  \hline
\endfirsthead
  \hline
$support_{chm}$ & $support_{tspec}$ & threshold & $R^2_{adj}$ & $rmse_{cv}$ & AIC & $R^2_{adj, ref}$ & $rmse_{cv, ref}$ & $AIC_{ref}$ \\ 
  \hline
\endhead
\hline
\multicolumn{9}{l}{\footnotesize Continued on next page}
\endfoot
\endlastfoot
ind & ind & 0\% & 0.46 & 134.23 & 64480.01 & 0.47 & 133.56 & 64480.01 \\ 
  ind & ind & 50\% & 0.46 & 134.42 & 64491.42 & 0.47 & 133.80 & 64491.42 \\ 
  ind & ind & 60\% & 0.46 & 134.39 & 64485.07 & 0.47 & 133.73 & 64485.07 \\ 
  ind & ind & 80\% & 0.47 & 133.90 & 64445.55 & 0.47 & 133.44 & 64445.55 \\ 
  ind & ind & 100\% & 0.46 & 134.15 & 64476.36 & 0.47 & 133.03 & 64476.36 \\ 
  ind & q25 & 0\% & 0.47 & 134.16 & 64829.60 & 0.47 & 133.46 & 64829.60 \\ 
  ind & q25 & 50\% & 0.46 & 134.40 & 64847.07 & 0.47 & 133.81 & 64847.07 \\ 
  ind & q25 & 60\% & 0.47 & 134.24 & 64831.79 & 0.47 & 133.85 & 64831.79 \\ 
  ind & q25 & 80\% & 0.47 & 133.68 & 64791.42 & 0.47 & 133.36 & 64791.42 \\ 
  ind & q25 & 100\% & 0.47 & 133.81 & 64796.95 & 0.48 & 133.02 & 64796.95 \\ 
  ind & q50 & 0\% & 0.47 & 133.96 & 65056.48 & 0.47 & 133.61 & 65056.48 \\ 
  ind & q50 & 50\% & 0.47 & 134.14 & 65070.88 & 0.47 & 133.79 & 65070.88 \\ 
  ind & q50 & 60\% & 0.46 & 134.27 & 65080.88 & 0.47 & 133.68 & 65080.88 \\ 
  ind & q50 & 80\% & 0.47 & 133.75 & 65038.93 & 0.47 & 133.32 & 65038.93 \\ 
  ind & q50 & 100\% & 0.47 & 133.41 & 65011.19 & 0.48 & 132.93 & 65011.19 \\ 
  ind & q80 & 0\% & 0.47 & 133.95 & 65287.34 & 0.47 & 133.50 & 65287.34 \\ 
  ind & q80 & 50\% & 0.47 & 134.16 & 65300.34 & 0.47 & 133.87 & 65300.34 \\ 
  ind & q80 & 60\% & 0.47 & 134.08 & 65293.82 & 0.47 & 133.87 & 65293.82 \\ 
  ind & q80 & 80\% & 0.47 & 133.77 & 65275.66 & 0.47 & 133.44 & 65275.66 \\ 
  ind & q80 & 100\% & 0.47 & 133.32 & 65246.11 & 0.48 & 132.98 & 65246.11 \\ 
  ind & q100 & 0\% & 0.47 & 133.53 & 65586.83 & 0.47 & 133.45 & 65586.83 \\ 
  ind & q100 & 50\% & 0.47 & 133.94 & 65618.19 & 0.47 & 133.75 & 65618.19 \\ 
  ind & q100 & 60\% & 0.47 & 133.70 & 65603.53 & 0.47 & 133.69 & 65603.53 \\ 
  ind & q100 & 80\% & 0.47 & 133.32 & 65576.24 & 0.47 & 133.32 & 65576.24 \\ 
  ind & q100 & 100\% & 0.48 & 132.92 & 65542.94 & 0.48 & 132.91 & 65542.94 \\ 
  q25 & ind & 0\% & 0.47 & 134.41 & 64780.16 & 0.47 & 133.73 & 64780.16 \\ 
  q25 & ind & 50\% & 0.47 & 134.61 & 64794.59 & 0.47 & 134.07 & 64794.59 \\ 
  q25 & ind & 60\% & 0.47 & 134.33 & 64777.60 & 0.47 & 134.02 & 64777.60 \\ 
  q25 & ind & 80\% & 0.47 & 133.93 & 64756.34 & 0.48 & 133.58 & 64756.34 \\ 
  q25 & ind & 100\% & 0.47 & 134.29 & 64791.96 & 0.48 & 133.13 & 64791.96 \\ 
  q25 & q25 & 0\% & 0.48 & 134.03 & 65504.85 & 0.48 & 133.37 & 65504.85 \\ 
  q25 & q25 & 50\% & 0.47 & 134.42 & 65534.10 & 0.48 & 133.76 & 65534.10 \\ 
  q25 & q25 & 60\% & 0.48 & 134.02 & 65502.59 & 0.48 & 133.71 & 65502.59 \\ 
  q25 & q25 & 80\% & 0.48 & 133.76 & 65475.16 & 0.48 & 133.39 & 65475.16 \\ 
  q25 & q25 & 100\% & 0.48 & 133.67 & 65467.58 & 0.49 & 132.97 & 65467.58 \\ 
  q25 & q50 & 0\% & 0.48 & 133.77 & 65744.39 & 0.48 & 133.19 & 65744.39 \\ 
  q25 & q50 & 50\% & 0.47 & 134.06 & 65767.28 & 0.48 & 133.40 & 65767.28 \\ 
  q25 & q50 & 60\% & 0.48 & 133.98 & 65763.30 & 0.48 & 133.31 & 65763.30 \\ 
  q25 & q50 & 80\% & 0.48 & 133.34 & 65715.59 & 0.48 & 133.00 & 65715.59 \\ 
  q25 & q50 & 100\% & 0.48 & 133.00 & 65694.99 & 0.49 & 132.40 & 65694.99 \\ 
  q25 & q80 & 0\% & 0.48 & 133.60 & 65960.24 & 0.48 & 133.26 & 65960.24 \\ 
  q25 & q80 & 50\% & 0.48 & 133.87 & 65973.16 & 0.48 & 133.55 & 65973.16 \\ 
  q25 & q80 & 60\% & 0.48 & 133.77 & 65963.19 & 0.48 & 133.53 & 65963.19 \\ 
  q25 & q80 & 80\% & 0.48 & 133.52 & 65950.49 & 0.48 & 133.18 & 65950.49 \\ 
  q25 & q80 & 100\% & 0.48 & 133.30 & 65934.08 & 0.49 & 132.75 & 65934.08 \\ 
  q25 & q100 & 0\% & 0.48 & 133.18 & 66271.97 & 0.48 & 133.13 & 66271.97 \\ 
  q25 & q100 & 50\% & 0.48 & 133.52 & 66295.64 & 0.48 & 133.39 & 66295.64 \\ 
  q25 & q100 & 60\% & 0.48 & 133.52 & 66290.29 & 0.48 & 133.41 & 66290.29 \\ 
  q25 & q100 & 80\% & 0.48 & 133.04 & 66266.45 & 0.48 & 133.01 & 66266.45 \\ 
  q25 & q100 & 100\% & 0.49 & 132.59 & 66221.95 & 0.49 & 132.58 & 66221.95 \\ 
  q50 & ind & 0\% & 0.47 & 133.79 & 64739.35 & 0.48 & 133.11 & 64739.35 \\ 
  q50 & ind & 50\% & 0.47 & 133.95 & 64745.34 & 0.48 & 133.44 & 64745.34 \\ 
  q50 & ind & 60\% & 0.47 & 133.88 & 64738.45 & 0.48 & 133.43 & 64738.45 \\ 
  q50 & ind & 80\% & 0.47 & 133.58 & 64730.79 & 0.48 & 133.11 & 64730.79 \\ 
  q50 & ind & 100\% & 0.47 & 133.54 & 64739.74 & 0.48 & 132.62 & 64739.74 \\ 
  q50 & q25 & 0\% & 0.48 & 133.26 & 65448.81 & 0.49 & 132.59 & 65448.81 \\ 
  q50 & q25 & 50\% & 0.48 & 133.75 & 65481.44 & 0.49 & 132.92 & 65481.44 \\ 
  q50 & q25 & 60\% & 0.48 & 133.43 & 65453.59 & 0.49 & 132.91 & 65453.59 \\ 
  q50 & q25 & 80\% & 0.48 & 133.21 & 65432.13 & 0.49 & 132.69 & 65432.13 \\ 
  q50 & q25 & 100\% & 0.49 & 132.99 & 65414.52 & 0.49 & 132.33 & 65414.52 \\ 
  q50 & q50 & 0\% & 0.48 & 133.16 & 65703.73 & 0.49 & 132.36 & 65703.73 \\ 
  q50 & q50 & 50\% & 0.48 & 133.31 & 65716.88 & 0.49 & 132.57 & 65716.88 \\ 
  q50 & q50 & 60\% & 0.48 & 133.26 & 65712.36 & 0.49 & 132.54 & 65712.36 \\ 
  q50 & q50 & 80\% & 0.48 & 132.81 & 65683.06 & 0.49 & 132.25 & 65683.06 \\ 
  q50 & q50 & 100\% & 0.49 & 132.07 & 65629.14 & 0.49 & 131.74 & 65629.14 \\ 
  q50 & q80 & 0\% & 0.48 & 132.97 & 65907.36 & 0.49 & 132.53 & 65907.36 \\ 
  q50 & q80 & 50\% & 0.48 & 133.24 & 65921.57 & 0.49 & 132.82 & 65921.57 \\ 
  q50 & q80 & 60\% & 0.48 & 133.13 & 65916.29 & 0.49 & 132.79 & 65916.29 \\ 
  q50 & q80 & 80\% & 0.48 & 133.03 & 65908.22 & 0.49 & 132.62 & 65908.22 \\ 
  q50 & q80 & 100\% & 0.49 & 132.72 & 65887.21 & 0.49 & 132.18 & 65887.21 \\ 
  q50 & q100 & 0\% & 0.49 & 132.60 & 66219.82 & 0.49 & 132.48 & 66219.82 \\ 
  q50 & q100 & 50\% & 0.49 & 132.78 & 66236.92 & 0.49 & 132.72 & 66236.92 \\ 
  q50 & q100 & 60\% & 0.49 & 132.81 & 66234.74 & 0.49 & 132.72 & 66234.74 \\ 
  q50 & q100 & 80\% & 0.49 & 132.55 & 66222.37 & 0.49 & 132.46 & 66222.37 \\ 
  q50 & q100 & 100\% & 0.49 & 132.06 & 66177.66 & 0.49 & 132.03 & 66177.66 \\ 
  q80 & ind & 0\% & 0.47 & 134.57 & 64803.25 & 0.47 & 133.82 & 64803.25 \\ 
  q80 & ind & 50\% & 0.47 & 134.71 & 64807.20 & 0.47 & 134.21 & 64807.20 \\ 
  q80 & ind & 60\% & 0.47 & 134.60 & 64797.25 & 0.47 & 134.20 & 64797.25 \\ 
  q80 & ind & 80\% & 0.47 & 134.16 & 64773.29 & 0.47 & 133.79 & 64773.29 \\ 
  q80 & ind & 100\% & 0.47 & 134.25 & 64795.59 & 0.48 & 133.26 & 64795.59 \\ 
  q80 & q25 & 0\% & 0.48 & 134.25 & 65515.53 & 0.48 & 133.44 & 65515.53 \\ 
  q80 & q25 & 50\% & 0.47 & 134.54 & 65537.15 & 0.48 & 133.78 & 65537.15 \\ 
  q80 & q25 & 60\% & 0.48 & 134.29 & 65517.64 & 0.48 & 133.76 & 65517.64 \\ 
  q80 & q25 & 80\% & 0.48 & 133.94 & 65484.54 & 0.48 & 133.53 & 65484.54 \\ 
  q80 & q25 & 100\% & 0.48 & 133.89 & 65481.53 & 0.49 & 133.04 & 65481.53 \\ 
  q80 & q50 & 0\% & 0.48 & 133.76 & 65751.41 & 0.48 & 133.11 & 65751.41 \\ 
  q80 & q50 & 50\% & 0.47 & 133.96 & 65768.65 & 0.48 & 133.40 & 65768.65 \\ 
  q80 & q50 & 60\% & 0.47 & 133.95 & 65766.20 & 0.48 & 133.39 & 65766.20 \\ 
  q80 & q50 & 80\% & 0.48 & 133.52 & 65734.51 & 0.48 & 133.06 & 65734.51 \\ 
  q80 & q50 & 100\% & 0.48 & 132.80 & 65683.87 & 0.49 & 132.45 & 65683.87 \\ 
  q80 & q80 & 0\% & 0.48 & 133.61 & 65964.09 & 0.48 & 133.27 & 65964.09 \\ 
  q80 & q80 & 50\% & 0.48 & 133.91 & 65986.91 & 0.48 & 133.60 & 65986.91 \\ 
  q80 & q80 & 60\% & 0.48 & 133.75 & 65973.94 & 0.48 & 133.60 & 65973.94 \\ 
  q80 & q80 & 80\% & 0.48 & 133.65 & 65964.87 & 0.48 & 133.33 & 65964.87 \\ 
  q80 & q80 & 100\% & 0.48 & 133.25 & 65931.80 & 0.49 & 132.82 & 65931.80 \\ 
  q80 & q100 & 0\% & 0.48 & 133.29 & 66282.17 & 0.48 & 133.22 & 66282.17 \\ 
  q80 & q100 & 50\% & 0.48 & 133.58 & 66307.14 & 0.48 & 133.51 & 66307.14 \\ 
  q80 & q100 & 60\% & 0.48 & 133.69 & 66308.78 & 0.48 & 133.51 & 66308.78 \\ 
  q80 & q100 & 80\% & 0.48 & 133.19 & 66281.98 & 0.48 & 133.16 & 66281.98 \\ 
  q80 & q100 & 100\% & 0.49 & 132.68 & 66234.70 & 0.49 & 132.64 & 66234.70 \\ 
  q100 & ind & 0\% & 0.40 & 142.64 & 65410.22 & 0.41 & 142.07 & 65410.22 \\ 
  q100 & ind & 50\% & 0.40 & 142.88 & 65422.90 & 0.40 & 142.38 & 65422.90 \\ 
  q100 & ind & 60\% & 0.40 & 142.72 & 65407.17 & 0.40 & 142.41 & 65407.17 \\ 
  q100 & ind & 80\% & 0.40 & 142.34 & 65386.95 & 0.41 & 142.00 & 65386.95 \\ 
  q100 & ind & 100\% & 0.40 & 142.91 & 65434.69 & 0.41 & 141.55 & 65434.69 \\ 
  q100 & q25 & 0\% & 0.41 & 142.73 & 66160.90 & 0.41 & 141.95 & 66160.90 \\ 
  q100 & q25 & 50\% & 0.40 & 143.01 & 66179.74 & 0.41 & 142.27 & 66179.74 \\ 
  q100 & q25 & 60\% & 0.41 & 142.73 & 66160.25 & 0.41 & 142.30 & 66160.25 \\ 
  q100 & q25 & 80\% & 0.41 & 142.68 & 66141.80 & 0.41 & 141.99 & 66141.80 \\ 
  q100 & q25 & 100\% & 0.41 & 142.34 & 66116.49 & 0.42 & 141.48 & 66116.49 \\ 
  q100 & q50 & 0\% & 0.41 & 142.42 & 66407.67 & 0.41 & 141.65 & 66407.67 \\ 
  q100 & q50 & 50\% & 0.40 & 142.65 & 66424.41 & 0.41 & 141.96 & 66424.41 \\ 
  q100 & q50 & 60\% & 0.40 & 142.62 & 66421.19 & 0.41 & 141.97 & 66421.19 \\ 
  q100 & q50 & 80\% & 0.41 & 142.29 & 66396.88 & 0.41 & 141.63 & 66396.88 \\ 
  q100 & q50 & 100\% & 0.41 & 141.58 & 66346.14 & 0.42 & 140.99 & 66346.14 \\ 
  q100 & q80 & 0\% & 0.41 & 142.10 & 66605.90 & 0.41 & 141.85 & 66605.90 \\ 
  q100 & q80 & 50\% & 0.41 & 142.34 & 66622.90 & 0.41 & 142.18 & 66622.90 \\ 
  q100 & q80 & 60\% & 0.41 & 142.23 & 66613.83 & 0.41 & 142.19 & 66613.83 \\ 
  q100 & q80 & 80\% & 0.41 & 142.07 & 66603.97 & 0.41 & 141.86 & 66603.97 \\ 
  q100 & q80 & 100\% & 0.42 & 141.53 & 66562.97 & 0.42 & 141.32 & 66562.97 \\ 
  q100 & q100 & 0\% & 0.41 & 141.98 & 66941.24 & 0.41 & 142.01 & 66941.24 \\ 
  q100 & q100 & 50\% & 0.41 & 142.27 & 66965.61 & 0.41 & 142.24 & 66965.61 \\ 
  q100 & q100 & 60\% & 0.41 & 142.08 & 66949.79 & 0.41 & 142.25 & 66949.79 \\ 
  q100 & q100 & 80\% & 0.41 & 141.67 & 66923.58 & 0.41 & 141.79 & 66923.58 \\ 
  q100 & q100 & 100\% & 0.42 & 141.24 & 66885.59 & 0.42 & 141.22 & 66885.59 \\ 
  \hline
\end{longtable}
\endgroup



%==================================================================================================================%
\end{document}