\section{Discussion}
\label{sec:Dis}

% ----------------------------------------------------------------------- %
% ----------------------------------------------------------------------- %
\subsection{Stratification according to ALS Acquisition Years and Tree Species}
\label{sec:strat_dis}

Incorporating the main tree species of a sample location in the timber volume regression model increased the model accuracy and revealed strong evidence for the existence of a tree species specific behavior concerning timber volume on the plot level. This result seems reasonable regarding the species specific taper functions on single-tree level applied in the \bwi{} \citep{kublin2003, kublin2013}. \added{
These findings also agree with those of \citet{latifi2012} who found an almost identical improvement in RMSE of 2\% when stratifying to broadleaf and coniferous tree species. The overall RMSE of their model  was however 10\% smaller than in our study. This might be due to a more heterogeneous dataset of much smaller sample size, but also because the temporal alignment between the auxiliary data acquisition and the terrestrial survey was much better. Additionally, the number of different tree species present in their dataset was lower than in our case and only comprised Scots pine, European beech and oak. The individual effects of spruce and Douglas fir indicated by our model also support the findings of \citet{breidenbach2008}, who found a higher percentage of coniferous trees in a sample plot to increase the timber volume predictions. This was not true for Scots pine and oak whose effects turned out to be very similar for our dataset. However, in our study the stratification according to the ALS acquisition years severely limited the flexibility of species-specific prediction functions and model interpretability.} In particular, using the ALS acquisition years as categorical variables led to highly unbalanced datasets when stratifying according to the main plot tree species. This prevented the use of further stratification variables such as bioclimatic growing regions due to confounding effects and consequent singularities in the design matrices. \added{It also implied an artificial increase in the number of parameters in the OLS regression model, which was however not regarded as critical with respect to overfitting issues due to the high amount of observations used for fitting the regression coefficients \citep[Ch. 15.1]{draper2014}.} A stratification to the ALS acquisition years however proved to be an effective means in accounting for the artificially introduced noise in the data caused by quality variations and the large time-lags between the remote sensing and terrestrial data. \added{It allowed for a model accuracy that was very close to those reported by \citet{maack2016} who conducted a very similar study in the German federal state of Baden-W\"urttemberg.} Model accuracies were also particularly higher in ALS acquisition year strata in which the data showed considerably less noise or were closer to the date of the terrestrial survey. \added{This effect was significantly reduced or even removed when merging several ALS acquisition year strata. Promising steps with respect to more up-to-date canopy height information have already been made, as the topographic survey institution of RLP is currently processing a canopy height model from aerial imagery acquisitions for 2011 and 2012 covering the entire federal state. These aerial photography acquisitions will in the future be conducted in a two-year period, allowing to derive up-to-date canopy height information in the framework of future forest inventories. For a smaller study area, \citet{kirchhoefer2017} have already demonstrated that similar model accuracies for German NFI data can be achieved using imagery-based canopy height models.}\par

Incorporating the calibrated tree species information further improved the model accuracy by \changed{0.03} in adjusted R$^2$. Compared to the simple model only containing ALS height metrics, including the ALS quality and calibrated tree species information increased the adjusted $R^2$ by \changed{0.12} in total. A differentiated evaluation of the final regression model revealed that the highest R$^2$-values were achieved within ALS acquisitions year strata close or identical with the year of the terrestrial survey, showing differences of up to \changed{0.3} between the R$^2$s. Also the gain in R$^2$ by including the tree species information was largest (i.e. \changed{0.07}) in combination with ALS information acquired in the year of the terrestrial inventory. These insights were particularly interesting with respect to the further use of the regression model for small area estimations. Small area estimators generally gain modeling strength by defining the prediction model $globally$ (i.e. using all data in the inventory area), and then applying the so-derived prediction model to a subset of observations located within the area of interest \citep{mandallaz2013a}. Consequently, the proposed stratification technique in the prediction model is expected to yield a gain in model accuracy and a reduction of the small area estimation errors if the small area domain mostly includes data from strata that have high within-strata model accuracies. \added{Findings of \citet{breidenbach2008} indicated that a further increase in model accuracies could possibly be achieved when incorporating these categorical variables as random rather than fixed-effects in linear mixed-effects models \citep{pinheirobates2000}. The reason we did not apply this family of models was that small area regression estimators subsequently applied in RLP \citep{mandallaz2013a, mandallaz2013b} require the internal models to be fitted by OLS technique.}\par


% ----------------------------------------------------------------------- %
% ----------------------------------------------------------------------- %
\subsection{Calibration of Tree Species Map Information}
\label{sec:calib_dis}
The accuracy assessment of the initially derived main plot species from the classification map revealed the presence of misclassifications that led to a decrease in model accuracy. \added{This is in agreement with the potential effects of erroneous explanatory variables discussed in \citet{carroll2006} and \citet{gustafson2003}, i.e. an increase of variability (noise) in the data that can increase the amount of unexplainable variance and thereby reduce the model accuracy.} One reason for the misclassifications were that the classification algorithm of \citet{stoffels2015} was exclusively trained in pure stands with the objective to predict the \textit{dominant tree species} of a forest stand. Thus, our requirements on the classification map differed considerably from the ones imposed by \citet{stoffels2015} and have to be considered as far more difficult to meet. Firstly, the reference data used in the accuracy assessment also included understory trees that were recorded in the \bwi{} sample. Secondly, determining an exact spatial validation unit for a sample location (support) is not possible due to the properties of angle count sampling (section \ref{sec:supp}). Thirdly, distinct discrepancies in the spatial scale between the reference data and the classification map severely hamper exact predictions of the main plot tree species especially in mixed forest stands. The latter issue caused a pronounced dependency of the user's accuracy on the support and threshold choice, particularly for tree species that most commonly occur in mixed forest structures, i.e. \textit{Scots pine} (91\%), $oak$ (90\%) and $beech$ (85\%) \citep{bwi3}. With respect to this set-up, the application of our calibration method proved to be of high value. It led to an increase in the classification accuracies, particularly for those tree species that performed worse in the uncalibrated setup, and thereby successfully minimized and even removed the deleterious effect of misclassifications on model accuracy and regression coefficients. Whereas the extensive analysis in our study deepened the understanding of the afore mentioned scale-effects, an alternative method for future applications could be to use map-derived percentages of each tree species as predictor variables in the random forest algorithm in order to directly predict the terrestrially observed main plot tree species.


% ----------------------------------------------------------------------- %
% ----------------------------------------------------------------------- %
\subsection{Choice of Support under Angle Count Sampling}
\label{sec:supp_choice_dis}

\changed{The validation of different support sizes underlined that the support choice can impact the accuracy of a prediction model, and thus confirmed the findings of \citet{deo2016}. In the present study, differences in the model accuracies however turned out to be small for most support choices. An exception was the choice of the $q100$ support for the CHM derived variables (76 meter radius), where the model accuracy was considerably worse than under the optimal settings. Contrary to our hypothesis, the use of plot-individual supports did not yield the best prediction performance overall. \citet{kirchhoefer2017} recently came to the same result when they transferred the angle-count sampling technique to a pixel-wise selection method of the auxiliary data that resembles the sample tree selection even more precisely. In their study, the application of fixed support sizes did also not perform worse than under variable supports. We consider two plausible reasons for the joint findings: first, the determination of an exact spatial extent that can be transferred to auxiliary data extraction remains technically infeasible under angle count sampling. Thus, angle count sampling does not seem to be adequate when linking inventory information with remote sensing data. Secondly, inaccuracies in the DGPS-measurements of the plot center locations as reported by \citet{lambrecht2017} may have an increased impact on the model accuracy the more exact the auxiliary data derivation spatially corresponds to those of the field survey. However, the extensive analysis carried out in our study also indicated that the optimal support size does not only depend on the spatial extent of the field plots, but also on the spatial resolution of the remote sensing data as well as the context in which the derived information is used in the prediction model. In the case of transforming the tree species information map into a suitable categorical predictor variable, the use of a large support size of 76 meter radius turned out to yield the best model accuracy. However, only few sample locations in the study area were actually characterized by limiting circles of that particular size. An analysis to find the best support settings therefore seems to be advisable prior to further applications of design-based or model-dependent inventory methods so as not to lose model accuracy by unsuitable support choices. The concept of the demonstrated analysis method for identifying suitable supports can be transferred to any kind of auxiliary information, predictor variable and prediction model.}