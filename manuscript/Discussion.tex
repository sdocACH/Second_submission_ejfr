\section{Discussion}
\label{sec:Dis}

% ----------------------------------------------------------------------- %
% ----------------------------------------------------------------------- %
\subsection{Stratification according to Tree Species and LiDAR Acquisitions}
\label{sec:strat_dis}

Incorporating the main tree species of a sample location in the timber volume regression model significantly increased the model accuracy and revealed strong evidence for the existence of a tree species specific behaviour concerning timber volume on the plot level. This result seems reasonable regarding the species specific taper functions on single-tree level applied in the \bwi{} \citep{kublin2003, kublin2013}. Further evidence and specification of the tree species effects on sample plot level - up to modeling individual tree species - would be desirable. However, this was not possible in our study because the stratification according to the LiDAR acquisition years severely limited the flexibility of species-specific prediction functions and model interpretability. In particular, using the LiDAR acquisition years as categorical variables led to highly unbalanced datasets when stratifying according to the main plot tree species, and prevented the use of further stratification variables such as bioclimatic growing regions due to confounding effects and consequent singularities in the design matrices. A stratification to the LiDAR acquisition years however proved to be a means in accounting for the artificially introduced noise in the data caused by quality variations and the large time-lags between the remote sensing and terrestrial data. Incorporating the calibrated tree species information further improved the model accuracy by 4\% in adjusted $R^2$. Compared to the simple model only containing LiDAR height metrics, including the LiDAR quality and calibrated tree species information increased the adjusted $R^2$ by 13\% in total. A differentiated evaluation of the final regression model revealed that the highest $R^2$ where achieved within LiDAR acquisitions year strata identical with the year of the terrestrial survey. Also the gain in the $R^2$ by including the tree species information was largest (i.e. 7\%) in combination with LiDAR information acquired in the year of the terrestrial inventory. These insights were particularly interesting with respect to the further use of the regression model for small area estimations. Small area estimators generally gain modeling strength by defining the prediction model $globally$ (i.e. using all data in the inventory area), and then applying the so-derived prediction model to a subset of observations located within the area of interest \citep{mandallaz2016}. Consequently, the proposed stratification technique in the prediction model could be expected to yield a gain in model accuracy and a reduction of the small area estimation errors if the small area domain mostly includes data from strata that have high within-strata model accuracies. This hypothesis is subject to ongoing analysis.

% ----------------------------------------------------------------------- %
% ----------------------------------------------------------------------- %
\subsection{Calibration of Tree Species Map Information}
\label{sec:calib_dis}
The accuracy assessment of the initially derived main plot species from the classification map revealed the presence of misclassifications that led to a decrease in model accuracy. One reason for the misclassifications were that the classification algorithm of \citet{stoffels2015} was exclusively trained in pure stands with the objective to predict the \textit{dominant tree species} of a forest stand. Thus, our requirements on the classification map differed considerably from the ones imposed by \citet{stoffels2015} and have to be considered as far more difficult to meet. Firstly, the reference data used in the accuracy assessment also included understory trees that were recorded in the \bwi{} sample. Secondly, determining an exact spatial validation unit for a sample location (support) is not possible due to the properties of angle count sampling (section \ref{sec:supp}). Thirdly, distinct discrepancies in the spatial scale between the reference data and the classification map severely hamper exact predictions of the main plot tree species especially in mixed forest stands. The latter issue caused a pronounced dependency of the user's accuracy on the support and threshold choice, particularly for tree species that most commonly occur in mixed forest structures, i.e. \textit{Scots pine} (91\%), $oak$ (90\%) and $beech$ (85\%) \citep{bwi3}. With respect to this set-up, the application of our calibration method proved to be of high value. It led to an increase in the classification accuracies, particularly for those tree species that performed worse in the uncalibrated setup, and thereby successfully minimized and even removed the deleterious effect of misclassifications on model accuracy and regression coefficients. We consider this \textit{a posteriori calibration} a valuable method for future studies where an external tree species map (i.e. the map was not created for the specific study objective) is used in prediction models. Whereas the extensive analysis in our study deepened the understanding of the afore mentioned scale-effects, an alternative method for future applications could be to use map-derived percentages of each tree species as predictor variables in the random forest algorithm in order to directly predict the terrestrially observed main plot tree species.

% ----------------------------------------------------------------------- %
% ----------------------------------------------------------------------- %
\subsection{Choice of Support under Angle Count Sampling}
\label{sec:supp_choice_dis}

The validation of different support sizes underlined that the support choice can impact prediction accuracy. In the present study, differences in the model accuracies turned out to be small for most support choices. An exception was the choice of the $q100$ support for the CHM derived variables (76 meter side length), where the model accuracy was considerably worse than what was achieved under optimal settings. With the exception of the latter, the accuracy differences according to \adjrsq{} and \rmsecv{} were very similar to those found by \citet{deo2016} when evaluating the model performance of optimal support sizes for a range of various basal area factors. An analysis to find the best support settings therefore seems to be advisable prior to further applications of model-assisted or model-dependent inventory methods so as not to lose model accuracy by unsuitable support choices. The concept of the demonstrated analysis method for identifying suitable supports can be transferred to any kind of auxiliary information, predictor variable and prediction model.\par
Contrary to our hypothesis, the use of plot-individual supports did not yield the best prediction performances. A plausible reason for this is that determining an exact plot radius under angle count sampling is technically infeasible, and thus, angle count sampling does not seem to be adequate when linking inventory information with remote sensing data. However, the extensive analysis carried out in our study indicated that the optimal support size depends on the spatial resolution of the remote sensing data as well as the context in which the derived information is used in the prediction model. In the case of transforming the tree species information map into a suitable categorical predictor variable, the use of a large support size of 76 meter side length turned out to yield the best model accuracy. However, only few sample locations in the study area were actually characterized by limiting circles of that particular size.
