
\section{Introduction}
\label{sec:intro}

Forest inventory methods are the primary tools used to assess the current state and development of forests over time. They provide reliable evidence-based information that is used to define and identify management actions as well as to adapt forest management strategies to both national and international guidelines. Two methods that have become particularly attractive are so-called \textit{double-sampling} \citep[\added{Ch. 5}]{mandallaz2008} and \textit{mapping} \changed{\citep{brosofske2014}} procedures. The core concept of these methods is to use predictions of the terrestrial target variable at additional sample locations where the terrestrial information has not been gathered. These predictions are produced by models that use explanatory variables derived from \textit{auxiliary data}, commonly in the form of spatially exhaustive remote sensing data in the inventory area. \added{Especially models to predict timber volume based on airborne laser scanning (ALS) have been extensively investigated for a long time \citep{naesset1997}.} The specific scope of double-sampling is to enlarge the terrestrial sample size by a much larger sample of predictions of the target variable in order to gain higher estimation precision without performing additional expensive terrestrial measurements. Model-dependent and design-based regression estimators are used in a broad range of double sampling concepts and methods \citep{gregoire2007, kohl2006, schreuder1993, saborowski2010, mandallaz2013a, mandallaz2013c} and have been applied to existing inventory systems \citep{breidenbach2012, vonLuebke2014, mandallaz2013b, magnussen2014, massey2014a}. While double-sampling methods provide reliable estimates for a given spatial unit, e.g. a forest district, they do not provide information about the spatial distribution of the estimated quantity within this area. For this reason, the same modeling technique used in double-sampling procedures has also been intensively used to produce exhaustive prediction maps that provide pixelwise estimations of a target variable in high spatial resolution \citep{bohlin2017, latifi2010, tonolli2011, hill2014, nink2015}.\par

To allow for an area-wide application of the prediction model, both double sampling and mapping methods require that the remote sensing data are available over the entire inventory area. This is usually not a limiting factor in \textit{small-scale} applications. In the optimal case, the remote sensing data are in principle collected in accordance to the specific study objective. Quality standards that have often been addressed are that \textit{a)} the remote sensing data should be acquired close to or even at the time of the terrestrial inventory in order to ensure best possible comparability between the target variable on the ground and the remote sensing derived variables \citep{mcroberts2015}; \textit{b)} the remote sensing technology and its spectral and spatial resolution should be chosen according to the modelling purpose \citep{kohl2006}; and \textit{c)} the variation in quality of the remote sensing data over the inventory area should be minimized in order to avoid artificial noise in the data \citep{naesset2014inmaltamo}. Despite the increasing availability and decreasing costs of remote sensing data \citep{white2016}, these quality standards of the remote sensing data can often not be guaranteed for \textit{large-scale} applications \citep{maack2016}, and trade-offs must be accepted \citep{jakubowski2013}. The prime objective is then to produce the best possible prediction model given the restrictions imposed by the available remote sensing information. The exploration of scarcely used remote sensing products and the optimization of prediction models under severe quality restrictions in the remote sensing data are thus one of the challenges in large-scale model-supported inventory applications.\par

Among the still rarely used remote sensing data in large scale applications, the integration of tree species information in prediction models - especially for timber volume estimation - has been stated as some of the most promising but often missing information \citep{koch2010, white2016}. As timber volume estimations on the single tree level in forest inventories are often based on species-specific biomass and volume equations \citep{husmann2017,zianis2005}, the application of species-specific models is expected to be a key factor for improving estimation precision \citep{white2016}. \changed{This has been supported by studies from \citet{breidenbach2008} who achieved a substantial improvement in accuracy of their timber volume prediction model when including a variable estimating the deciduous proportion derived from leaf-off ALS data. Similar gains in model performance were also reported by \citet{straub2009} and \citet{latifi2012} who used broadleaf and coniferous information based on color infrared orthophotos as a categorical explanatory variable. However, studies that explore the use of more species-specific information (i.e. a further discrimination of tree species) as explanatory variables have been rare. Further investigations are thus necessary especially in countries whose forests are characterized by a larger variety of tree species that may also occur in mixed and uneven-aged stands \citep{mcroberts2010}.} The area-wide tree species information in most studies was obtained from satellite and airborne remote sensing sensors based on automatic classification methods. Whereas the presence of misclassifications has already been addressed \citep{latifi2012}, an issue that has so far been neglected is how misclassifications actually affect the prediction model \citep{gustafson2003}.\par

A frequently encountered problem in large scale forest inventories is the lack of temporal synchronicity between the remote sensing acquisition and the terrestrial survey. As a result, the available remote sensing data often exhibit notable time-lags with respect to the date of the terrestrial inventory. This has often been addressed as a major drawback, especially for the application of design-based change estimation \citep{massey2015b}.\par

Our study is embedded in the current implementation of design-based regression estimators \citep{mandallaz2013a, mandallaz2013b, mandallaz2013c} for estimating the standing timber volume within the state and communal forest management units over the entire state of Rhineland-Palatinate (RLP, Germany). With respect to this overall objective, the aim of this study was to derive an ordinary least square (OLS) regression model to generate predictions of the standing timber volume associated with a sample location of the Third German National Forest Inventory (BWI3) over the entire state and communal forest area (6155 km$^2$). A merged ALS dataset from different acquisition years and a satellite-based tree species classification map for the five main tree species in RLP was available for the entire inventory area and consequently used to derive predictor variables. The major limiting factors for using these data in a regression analysis are \textbf{(i)} variation in the ALS data quality as well as time-lags of up to 10 years between the ALS acquisitions and the terrestrial survey, \textbf{(ii)} misclassifications in the tree species classification map and \textbf{(iii)} the ambiguous choice of a suitable extraction area (\textit{support}) for all remote sensing information under angle count sampling in the terrestrial survey (variable sample plot sizes). For this reason, we address the following specific research questions:

\begin{enumerate}
 \item How can tree species map information be optimally used within a regression model that predicts timber volume? What effects do misclassifications have on the predictions and how can these effects be minimized?
 \item What are the effects of quality restrictions and substantial time lags between the ALS- and terrestrial data acquisition on the regression model and how can these effects be mitigated?
 \item Does support size influence model accuracy? What is the optimal support size and what are the determining factors?
\end{enumerate}