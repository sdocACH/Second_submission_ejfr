\section{Conclusion}
\label{sec:concl}

% Core Conclusions:
We draw three major conclusions from our study: (1) our analyses strongly indicated that the acquisition of auxiliary information close to the date of the terrestrial survey is a key factor to achieve good model accuracies. Particularly for large-scale inventory applications, this requirement is often difficult to meet. In such cases, we consider that the proposed method of including quality information about the auxiliary data in a prediction model an effective technique for improving the prediction accuracy. It will also be interesting to investigate how estimation accuracy of model-assisted estimators can be improved by the modeling technique. (2) Our study also indicated that the relationship between field measured timber volume and remote-sensing derived height information is tree species specific. We here expect the tree species information to become even more predictive when combined with explanatory variables such as bioclimatic growing conditions, soil properties and stand density on the plot level. Whereas this was not feasible in our study due to the unbalancedness of the data set, testing this hypothesis will be possible in the near future as promising steps with respect to more up-to-date height information have already been made. (3) We consider the demonstrated \textit{a posteriori} calibration technique to be a valuable method for future studies where an external tree species map (i.e. the map was not created for the specific study objective) is used in prediction models. The application of a calibration-model can however be extended to any error-prone explanatory variable and be a simple means to clean the data set from noise and thus increase the model accuracy.\\

% Our analyses strongly indicate that the acquisition of the auxiliary information close to the date of the terrestrial survey is a key factor in order to increase the model accuracy. We also expect the tree species information in the timber volume model to become even more relevant if the temporal synchronocity and the quality of the canopy height information is improved.
%
%Promising steps with respect to more up-to-date auxiliary information have already been made, as the topographic survey institution of RLP is currently processing a canopy height model from aerial imagery acquisitions for 2011 and 2012 covering the entire federal state. These aerial photography acquisitions will in the future be conducted in a two-year period, allowing to derive up-to-date canopy height information in the framework of future forest inventories. Investigating the performance between aerial and ALS derived canopy height models and their consequent predictive power in the frame of timber volume estimations are tasks for subsequent analysis. 
%
%Additionally, availability of satellite data for tree species classification map production with respect to up-to-dateness and coverage has recently been increasing in the frame of the Sentinel-2 mission \citep{sentinel2}

% old version:
The objective of this study was to identify a suitable ordinary least square regression model that can be applied over the entire forest area of Rhineland-Palatinate using model-assisted estimators. The large amount of data that was gathered in the frame of this study allowed for extensive modeling possibilities, but had the side effect of contributing to high heterogeneity in the response and explanatory variables. Whereas the variability of the response variable (timber volume on plot level) is due to the very heterogeneous forest structures and bioclimatic growing regions in RLP, a considerable amount of heterogeneity in the explanatory variables was introduced by quality restrictions in the remote sensing data. This was particularly true for the ALS derived canopy height information that was gathered in a time span of ten years around the date of the terrestrial inventory and revealed pronounced quality variations. With an \adjrsq{} of 0.48 and a \rmsecv{} of 140 \mha{} (46.7\%), the model accuracy was still very close to those found in similar studies \citep{maack2016}. Our analyses strongly indicate that the acquisition of the auxiliary information close to the date of the terrestrial survey is a key factor in order to increase the model accuracy. We also expect the tree species information in the timber volume model to become even more relevant if the temporal synchronocity and the quality of the canopy height information is improved. An up-to-date canopy height model would also circumvent a stratification according to different ALS acquisition characteristics, lead to a more balanced dataset when stratifying for the main plot tree species and allow for incorporating information that can further explain the variation within each tree species group. With respect to the latter, information about the bioclimatic growing conditions, soil properties and the stand density on plot level are expected to further improve the model's predictive performance. Promising steps with respect to more up-to-date auxiliary information have already been made, as the topographic survey institution of RLP is currently processing a canopy height model from aerial imagery acquisitions for 2011 and 2012 covering the entire federal state. These aerial photography acquisitions will in the future be conducted in a two-year period, allowing to derive up-to-date canopy height information in the framework of future forest inventories. As the availability of countrywide imagery-based surface models has been increasing \citep{ginzler2015}, investigating the performance between aerial and ALS derived canopy height models and their consequent predictive power in the frame of timber volume estimations \citep{ullah2017} are tasks for subsequent analysis. Additionally, availability of satellite data for tree species classification map production with respect to up-to-dateness and coverage has recently been increasing in the frame of the Sentinel-2 mission \citep{sentinel2}.